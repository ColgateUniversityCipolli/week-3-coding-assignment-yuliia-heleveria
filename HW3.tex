\documentclass{article}\usepackage[]{graphicx}\usepackage[]{xcolor}
% maxwidth is the original width if it is less than linewidth
% otherwise use linewidth (to make sure the graphics do not exceed the margin)
\makeatletter
\def\maxwidth{ %
  \ifdim\Gin@nat@width>\linewidth
    \linewidth
  \else
    \Gin@nat@width
  \fi
}
\makeatother

\definecolor{fgcolor}{rgb}{0.345, 0.345, 0.345}
\newcommand{\hlnum}[1]{\textcolor[rgb]{0.686,0.059,0.569}{#1}}%
\newcommand{\hlsng}[1]{\textcolor[rgb]{0.192,0.494,0.8}{#1}}%
\newcommand{\hlcom}[1]{\textcolor[rgb]{0.678,0.584,0.686}{\textit{#1}}}%
\newcommand{\hlopt}[1]{\textcolor[rgb]{0,0,0}{#1}}%
\newcommand{\hldef}[1]{\textcolor[rgb]{0.345,0.345,0.345}{#1}}%
\newcommand{\hlkwa}[1]{\textcolor[rgb]{0.161,0.373,0.58}{\textbf{#1}}}%
\newcommand{\hlkwb}[1]{\textcolor[rgb]{0.69,0.353,0.396}{#1}}%
\newcommand{\hlkwc}[1]{\textcolor[rgb]{0.333,0.667,0.333}{#1}}%
\newcommand{\hlkwd}[1]{\textcolor[rgb]{0.737,0.353,0.396}{\textbf{#1}}}%
\let\hlipl\hlkwb

\usepackage{framed}
\makeatletter
\newenvironment{kframe}{%
 \def\at@end@of@kframe{}%
 \ifinner\ifhmode%
  \def\at@end@of@kframe{\end{minipage}}%
  \begin{minipage}{\columnwidth}%
 \fi\fi%
 \def\FrameCommand##1{\hskip\@totalleftmargin \hskip-\fboxsep
 \colorbox{shadecolor}{##1}\hskip-\fboxsep
     % There is no \\@totalrightmargin, so:
     \hskip-\linewidth \hskip-\@totalleftmargin \hskip\columnwidth}%
 \MakeFramed {\advance\hsize-\width
   \@totalleftmargin\z@ \linewidth\hsize
   \@setminipage}}%
 {\par\unskip\endMakeFramed%
 \at@end@of@kframe}
\makeatother

\definecolor{shadecolor}{rgb}{.97, .97, .97}
\definecolor{messagecolor}{rgb}{0, 0, 0}
\definecolor{warningcolor}{rgb}{1, 0, 1}
\definecolor{errorcolor}{rgb}{1, 0, 0}
\newenvironment{knitrout}{}{} % an empty environment to be redefined in TeX

\usepackage{alltt}
\usepackage[margin=1.0in]{geometry} % To set margins
\usepackage{amsmath}  % This allows me to use the align functionality.
                      % If you find yourself trying to replicate
                      % something you found online, ensure you're
                      % loading the necessary packages!
\usepackage{amsfonts} % Math font
\usepackage{fancyvrb}
\usepackage{hyperref} % For including hyperlinks
\usepackage[shortlabels]{enumitem}% For enumerated lists with labels specified
                                  % We had to run tlmgr_install("enumitem") in R
\usepackage{float}    % For telling R where to put a table/figure
\usepackage{natbib}        %For the bibliography
\bibliographystyle{apalike}%For the bibliography
\IfFileExists{upquote.sty}{\usepackage{upquote}}{}
\begin{document}

\begin{enumerate}
%%%%%%%%%%%%%%%%%%%%%%%%%%%%%%%%%%%%%%%%%%%%%%%%%%%%%%%%%%%%%%%%%%%%%%%%%%%%%%%%
%%%%%%%%%%%%%%%%%%%%%%%%%%%%%%%%%%%%%%%%%%%%%%%%%%%%%%%%%%%%%%%%%%%%%%%%%%%%%%%%
% QUESTION 1
%%%%%%%%%%%%%%%%%%%%%%%%%%%%%%%%%%%%%%%%%%%%%%%%%%%%%%%%%%%%%%%%%%%%%%%%%%%%%%%%
%%%%%%%%%%%%%%%%%%%%%%%%%%%%%%%%%%%%%%%%%%%%%%%%%%%%%%%%%%%%%%%%%%%%%%%%%%%%%%%%
\item This week's Problem of the Week in Math is described as follows:
\begin{quotation}
  \textit{There are thirty positive integers less than 100 that share a certain 
  property. Your friend, Blake, wrote them down in the table to the left. But 
  Blake made a mistake! One of the numbers listed is wrong and should be replaced 
  with another. Which number is incorrect, what should it be replaced with, and 
  why?}
\end{quotation}
The numbers are listed below.
\begin{center}
  \begin{tabular}{ccccc}
    6 & 10 & 14 & 15 & 21\\
    22 & 26 & 33 & 34 & 35\\
    38 & 39 & 46 & 51 & 55\\
    57 & 58 & 62 & 65 & 69\\
    75 & 77 & 82 & 85 & 86\\
    87 & 91 & 93 & 94 & 95
  \end{tabular}
\end{center}
Use the fact that the ``certain'' property is that these numbers are all supposed
to be the product of \emph{unique} prime numbers to find and fix the mistake that
Blake made.\\
\textbf{Reminder:} Code your solution in an \texttt{R} script and copy it over
to this \texttt{.Rnw} file.\\
\textbf{Hint:} You may find the \verb|%in%| operator and the \verb|setdiff()| function to be helpful.\\

\textbf{Solution:} 
% Write your answer and explanations here.

\begin{knitrout}\scriptsize
\definecolor{shadecolor}{rgb}{0.969, 0.969, 0.969}\color{fgcolor}\begin{kframe}
\begin{alltt}
\hlcom{#needed package:}
\hlcom{#install.packages("numbers")}
\hlkwd{library}\hldef{(}\hlsng{"numbers"}\hldef{)}

\hlcom{#list of numbers to check}
\hldef{provided.nums} \hlkwb{<-} \hlkwd{c}\hldef{(}\hlnum{6}\hldef{,} \hlnum{10}\hldef{,} \hlnum{14}\hldef{,} \hlnum{15}\hldef{,} \hlnum{21}\hldef{,}
                   \hlnum{22}\hldef{,} \hlnum{26}\hldef{,} \hlnum{33}\hldef{,} \hlnum{34}\hldef{,} \hlnum{35}\hldef{,}
                   \hlnum{38}\hldef{,} \hlnum{39}\hldef{,} \hlnum{46}\hldef{,} \hlnum{51}\hldef{,} \hlnum{55}\hldef{,}
                   \hlnum{57}\hldef{,} \hlnum{58}\hldef{,} \hlnum{62}\hldef{,} \hlnum{65}\hldef{,} \hlnum{69}\hldef{,}
                   \hlnum{75}\hldef{,} \hlnum{77}\hldef{,} \hlnum{82}\hldef{,} \hlnum{85}\hldef{,} \hlnum{86}\hldef{,}
                   \hlnum{87}\hldef{,} \hlnum{91}\hldef{,} \hlnum{93}\hldef{,} \hlnum{94}\hldef{,} \hlnum{95}\hldef{)}

\hlcom{#Function to determine which numbers are products of unique primes}
\hlcom{#Function takes in a number and returns true if num is a product of unique}
\hlcom{#primes and false otherwise}
\hldef{get_valid_numbers} \hlkwb{<-} \hlkwa{function}\hldef{(}\hlkwc{num}\hldef{)\{}
  \hldef{factors} \hlkwb{<-} \hlkwd{primeFactors}\hldef{(num)} \hlcom{#get prmime factors of the num}
  \hldef{unique.factors} \hlkwb{<-} \hlkwd{unique}\hldef{(factors)} \hlcom{#get unique factors}
  \hlkwa{if} \hldef{(}\hlkwd{length}\hldef{(unique.factors)} \hlopt{==}\hlnum{1}\hldef{)\{} \hlcom{#if the number contains only 1 prime}
    \hlkwd{return}\hldef{(}\hlnum{FALSE}\hldef{)}
  \hldef{\}}
  \hlkwd{return}\hldef{(}\hlkwd{length}\hldef{(factors)} \hlopt{==} \hlkwd{length}\hldef{(unique.factors))}
\hldef{\}}

\hldef{num.to.check} \hlkwb{<-} \hlkwd{c}\hldef{(}\hlnum{1}\hlopt{:}\hlnum{99}\hldef{)}
\hldef{valid.nums} \hlkwb{<-} \hlkwd{c}\hldef{()}
\hlcom{#check every number if it has unique factors}
\hlcom{#if true, add to the list of valid numbers}
\hlkwa{for} \hldef{(i} \hlkwa{in} \hlnum{1}\hlopt{:}\hlkwd{length}\hldef{(num.to.check))\{}
  \hlkwa{if} \hldef{(}\hlkwd{get_valid_numbers}\hldef{(num.to.check[i]))\{}
    \hldef{valid.nums} \hlkwb{<-} \hlkwd{c}\hldef{(valid.nums, num.to.check[i])}
  \hldef{\}}
\hldef{\}}
\hlcom{#numbers that are valid but not in provided list}
\hldef{valid.not.included.nums} \hlkwb{<-} \hlkwd{setdiff}\hldef{(valid.nums, provided.nums)}
\hlcom{#numbers that are in provided list but not valid}
\hldef{wrong.num} \hlkwb{<-} \hlkwd{setdiff}\hldef{(provided.nums, valid.nums)}

\hlcom{#get index of the wrong number}
\hldef{wrong.index} \hlkwb{<-} \hlkwd{which}\hldef{(provided.nums}\hlopt{==}\hldef{wrong.num)}
\hlcom{#get the number after wrong }
\hldef{next.index} \hlkwb{<-} \hldef{wrong.index}\hlopt{+}\hlnum{1}
\hldef{next.after.wrong} \hlkwb{=} \hlnum{99}
\hlkwa{if} \hldef{(next.index} \hlopt{<=} \hlkwd{length}\hldef{(provided.nums))\{} \hlcom{#check bounds}
  \hldef{next.after.wrong} \hlkwb{=} \hldef{provided.nums[next.index]}
\hldef{\}}
\hlcom{#get the number before wrong}
\hldef{previous.index} \hlkwb{<-} \hldef{wrong.index}\hlopt{-}\hlnum{1}
\hldef{prev.before.wrong} \hlkwb{=} \hlnum{1}
\hlkwa{if} \hldef{(previous.index} \hlopt{>} \hlnum{0}\hldef{)\{}
  \hldef{prev.before.wrong} \hlkwb{=} \hldef{provided.nums[previous.index]}
\hldef{\}}
\hldef{current.to.replace} \hlkwb{=} \hlnum{0} \hlcom{#placeholder for the number that will replace the wrong one}
\hlcom{#replace the wrong number with a valid product of unique prime numbes}
\hlkwa{for} \hldef{(i} \hlkwa{in} \hlnum{1}\hlopt{:}\hlkwd{length}\hldef{(valid.not.included.nums))\{}
  \hlcom{#iterate the valid numbers until we find the one in bounds of previous and next}
  \hlkwa{if} \hldef{(valid.not.included.nums[i]} \hlopt{>} \hldef{prev.before.wrong} \hlopt{&&} \hldef{valid.not.included.nums[i]} \hlopt{<} \hldef{next.after.wrong)\{}
    \hldef{current.to.replace} \hlkwb{=} \hldef{valid.not.included.nums[i]}
  \hldef{\}}
\hldef{\}}
\hlcom{#replace the wrong number with the correct one}
\hldef{provided.nums[wrong.index]} \hlkwb{=} \hldef{current.to.replace}
\end{alltt}
\end{kframe}
\end{knitrout}
The wrong number was 75 and it was replaced with 74.

Correct list of numbers: 6, 10, 14, 15, 21, 22, 26, 33, 34, 35, 38, 39, 46, 51, 55, 57, 58, 62, 65, 69, 74, 77, 82, 85, 86, 87, 91, 93, 94, 95.
\end{enumerate}

\bibliography{bibliography}
\end{document}
